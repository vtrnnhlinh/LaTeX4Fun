%%%%%%%%%%%%%%%%%%%%%%%%%%%%%%%%%%%%%%%%%
% Long letter 长安大学版 Chang'an University Version
% Version 1.0 (2024-01-23)

% This template was revised by Shuaiming Chen(chenshuaiming@chd.edu.com) based on Zheng-hu Nie(brian.nie@gmail.com).

%%%%%%%%%%%%%%%%%%%%%%%%%%%%%%%%%%%%%%%%%

%----------------------------------------------------------------------------------------
%	PACKAGES AND OTHER DOCUMENT CONFIGURATIONS 文档基础配置
%----------------------------------------------------------------------------------------

\documentclass{article}

\usepackage{charter} % Use the Charter font
\usepackage[utf8]{vietnam}
\usepackage[
	a4paper, % Paper size
	top=1in, % Top margin
	bottom=1in, % Bottom margin
	left=1in, % Left margin
	right=1in, % Right margin
	%showframe % Uncomment to show frames around the margins for debugging purposes
]{geometry}

\setlength{\parindent}{0pt} % Paragraph indentation
\setlength{\parskip}{1em} % Vertical space between paragraphs

\usepackage{graphicx} % Required for including images

\usepackage{fancyhdr} % Required for customizing headers and footers

\fancypagestyle{firstpage}{%
	\fancyhf{} % Clear default headers/footers
	\renewcommand{\headrulewidth}{0pt} % No header rule
	\renewcommand{\footrulewidth}{1pt} % Footer rule thickness
}

\fancypagestyle{subsequentpages}{%
	\fancyhf{} % Clear default headers/footers
	\renewcommand{\headrulewidth}{1pt} % Header rule thickness
	\renewcommand{\footrulewidth}{1pt} % Footer rule thickness
}

\AtBeginDocument{\thispagestyle{firstpage}} % Use the first page headers/footers style on the first page
\pagestyle{subsequentpages} % Use the subsequent pages headers/footers style on subsequent pages

%----------------------------------------------------------------------------------------

\begin{document}

%----------------------------------------------------------------------------------------
%	FIRST PAGE HEADER
%----------------------------------------------------------------------------------------

\includegraphics[width=0.35\textwidth]{bk_name_en.png} % Logo

\vspace{-1em} % Pull the rule closer to the logo

\rule{\linewidth}{1pt} % Horizontal rule

\bigskip\bigskip % Vertical whitespace

%----------------------------------------------------------------------------------------
%	YOUR NAME AND CONTACT INFORMATION
%----------------------------------------------------------------------------------------

\hfill
\begin{tabular}{l @{}}
\hfill April 2, 2024 \bigskip\\ % Date
\hfill Võ Trần Nhã Linh \\
\hfill Email: linh.vo1011@hcmut.edu.vn \\
\hfill Dorm A, VNU-HCM\\
\hfill Thu Duc City, Ho Chi Minh City, Vietnam \\ % Address
\end{tabular}

\bigskip % Vertical whitespace

%----------------------------------------------------------------------------------------
%	ADDRESSEE AND GREETING
%----------------------------------------------------------------------------------------

\begin{tabular}{@{} l}
	HPQC Team \\
	HPCC, HCMUT, VNU-HCM \\
	268 Ly Thuong Kiet, District 10, Ho Chi Minh City, Vietnam \\
\end{tabular}

\bigskip % Vertical whitespace

Dear HPQC Team Professors,

\bigskip % Vertical whitespace

%----------------------------------------------------------------------------------------
%	LETTER CONTENT
%----------------------------------------------------------------------------------------



I am writing to express my interests in joining the High Performance Quantumn Computing (HPQC) Team at HCMUT. I hope to have a chance to work and learn with brilliant minds in the field.

First thing I want to say that my academic profile isn't impressive. I have a pretty low GPA (2.2/4 after 231) and I have no research experience. Honestly, I have to encourage myself a lot to apply because I know I am pretty underqualified for this position. However, I have a dream for Quantumn Computing since high school after watching a video about it on TED-ed. It's also the reason why I choose Computer Engineering major (it was a naive idea, I know). But when entering university, I feel like Quantumn Computing is something very far away from my reach. I have no idea how to start, where to start, and what to do. I was lost, I have to keep the idea in a corner of my heart and choose a more practical path. So in my CV, you can see that I have a focus on IoT and Embedded Systems fields.

When I see the announcement on HPC Lab Page on Facebook, I feel very excited. I love sciences, specifically mathematics and physics. I love the idea to play with quantumn algorithms and pioneering ideas that can reshape Computer Science. The idea of working with limited resources and have a lot of constraints is the reason why I choose IoT and Embedded Systems fields after realizing I couldn't pursue Quantumn Computing. I am not that good but I love challenging problems and discovering things I don't know.

I am willing to learn and work hard to catch up with the team. I have enthusiastic heart, open mind and strong English proficiency. I hope that misters can give me a chance.

\bigskip % Vertical whitespace

Sincerely yours,

Linh
\vspace{20pt} % Vertical whitespace

Võ Trần Nhã Linh

\end{document}
